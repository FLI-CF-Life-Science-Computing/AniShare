%% Generated by Sphinx.
\def\sphinxdocclass{report}
\documentclass[letterpaper,10pt,openany,oneside,english]{sphinxmanual}
\ifdefined\pdfpxdimen
   \let\sphinxpxdimen\pdfpxdimen\else\newdimen\sphinxpxdimen
\fi \sphinxpxdimen=.75bp\relax

\PassOptionsToPackage{warn}{textcomp}
\usepackage[utf8]{inputenc}
\ifdefined\DeclareUnicodeCharacter
 \ifdefined\DeclareUnicodeCharacterAsOptional
  \DeclareUnicodeCharacter{"00A0}{\nobreakspace}
  \DeclareUnicodeCharacter{"2500}{\sphinxunichar{2500}}
  \DeclareUnicodeCharacter{"2502}{\sphinxunichar{2502}}
  \DeclareUnicodeCharacter{"2514}{\sphinxunichar{2514}}
  \DeclareUnicodeCharacter{"251C}{\sphinxunichar{251C}}
  \DeclareUnicodeCharacter{"2572}{\textbackslash}
 \else
  \DeclareUnicodeCharacter{00A0}{\nobreakspace}
  \DeclareUnicodeCharacter{2500}{\sphinxunichar{2500}}
  \DeclareUnicodeCharacter{2502}{\sphinxunichar{2502}}
  \DeclareUnicodeCharacter{2514}{\sphinxunichar{2514}}
  \DeclareUnicodeCharacter{251C}{\sphinxunichar{251C}}
  \DeclareUnicodeCharacter{2572}{\textbackslash}
 \fi
\fi
\usepackage{cmap}
\usepackage[T1]{fontenc}
\usepackage{amsmath,amssymb,amstext}
\usepackage[english]{babel}
\usepackage{times}
\usepackage[Lenny]{fncychap}
\usepackage{sphinx}

\usepackage{geometry}

% Include hyperref last.
\usepackage{hyperref}
% Fix anchor placement for figures with captions.
\usepackage{hypcap}% it must be loaded after hyperref.
% Set up styles of URL: it should be placed after hyperref.
\urlstyle{same}
\addto\captionsenglish{\renewcommand{\contentsname}{Contents:}}

\addto\captionsenglish{\renewcommand{\figurename}{Fig.}}
\addto\captionsenglish{\renewcommand{\tablename}{Table}}
\addto\captionsenglish{\renewcommand{\literalblockname}{Listing}}

\addto\captionsenglish{\renewcommand{\literalblockcontinuedname}{continued from previous page}}
\addto\captionsenglish{\renewcommand{\literalblockcontinuesname}{continues on next page}}

\addto\extrasenglish{\def\pageautorefname{page}}

\setcounter{tocdepth}{2}



\title{AniShare Documentation}
\date{Dec 07, 2018}
\release{1.6}
\author{Holger Dinkel, Fabian Monheim}
\newcommand{\sphinxlogo}{\vbox{}}
\renewcommand{\releasename}{Release}
\makeindex

\begin{document}

\maketitle
\sphinxtableofcontents
\phantomsection\label{\detokenize{index::doc}}



\chapter{Introduction}
\label{\detokenize{index:introduction}}
\sphinxstylestrong{anishare} is a webservice for research institutes to share animals with the goal to re-use
animals and thus minimize total animal usage.

It has been developed at the \sphinxhref{http://www.leibniz-fli.de}{Leibniz institute for aging research} in
Jena. This django app is meant to be used by researchers who want to share research animals with
their colleagues. The basic idea is that animals are bred for experiments; however, sometimes, not
all parts of the animal are used or sometimes an experiment gets cancelled for whatever reason. By
sharing animals within the institute, less animals in total have to be sacrificed for research.

Anishare is a simple database of animals offered for reuse and a easy way to claim an animal with
automatic generation of email messages as well as an RSS feed for updates.

\noindent\sphinxincludegraphics[width=0.990\linewidth]{{anishare_index}.png}

At the moment, the software/database is geared towards handling of mice, however, it can be adjusted
to handle any kind of research animal. AniShare is not connected to \sphinxtitleref{PyRat} or \sphinxtitleref{Tick@lab}.
The official changes (e.g. new ownership must be applied in \sphinxtitleref{PyRat} or \sphinxtitleref{Tick@lab} separately by the responsible person, in communication with the animal care takers).

This documentation can also be downloaded as pdf file: \sphinxhref{https://anishare.leibniz-fli.de/static/doc/anishare\_documentation.pdf}{Anishare Documentation}


\chapter{Contact}
\label{\detokenize{index:contact}}
Technical and application support: Fabian Monheim (CF Scientific IT), \sphinxhref{mailto:fabian.monheim@leibniz-fli.de}{fabian.monheim@leibniz-fli.de}, 03641-65-6872

Content support: Animal Facility and Animal Welfare Officer


\chapter{Licence}
\label{\detokenize{index:licence}}
The software was developed at the Leibniz Institute on Aging - Fritz Lipmann Institute (FLI; \sphinxurl{http://www.leibniz-fli.de/}) under a mixed licensing model.
Researchers at academic and non-profit organizations can use anishare using the included license, while for-profit organizations are required to purchase a license.
By downloading the package you agree with conditions of the FLI Software License Agreement for Academic Non-commercial Research (LICENSE.pdf).


\chapter{Sitemap}
\label{\detokenize{index:sitemap}}\begin{itemize}
\item {} 
Start: \sphinxurl{https://anishare.leibniz-fli.de}

\item {} 
Animal: \sphinxurl{https://anishare.leibniz-fli.de/animals}

\item {} 
Organ: \sphinxurl{https://anishare.leibniz-fli.de/organs}

\item {} 
Animal and organ feed: \sphinxurl{https://anishare.leibniz-fli.de/animals/feed}

\item {} 
Administration: \sphinxurl{https://anishare.leibniz-fli.de/admin}

\item {} 
Macros: \sphinxurl{https://anishare.leibniz-fli.de/macros}

\item {} 
Change history: \sphinxurl{https://anishare.leibniz-fli.de/changehistory}

\item {} 
Change history feed: \sphinxurl{https://anishare.leibniz-fli.de/changehistory/feed}

\end{itemize}


\chapter{User types}
\label{\detokenize{index:user-types}}\begin{itemize}
\item {} 
User: every FLI employee who wants to claim an animal.

\item {} 
Manager: this person is appointed within the research group and coordinates the offering/sharing of animals.

\item {} 
Person who perform euthanasia: this person will be named in anishare by the manager (relevant only for organ sharing).

\item {} 
Superuser: this person is administrator of the database and has the full control of the function (IT, animal welfare officers, veterinarians and heads of animal facilities).

\end{itemize}


\chapter{Main user interface}
\label{\detokenize{index:main-user-interface}}

\section{Animals}
\label{\detokenize{index:animals}}
The main user-facing site is the list of animals to be shared. A user can browse this list, sort it
via the headers or search for a term using the search bar.

\noindent\sphinxincludegraphics[width=0.990\linewidth]{{anishare_index}.png}

If a user is interested in an animal, they should click on the button “Claim” which will bring up
another page (see below) in which they can review their claim before finally submitting. When they
click on “Yes, I want to claim this!”, then they will be entered as \sphinxstyleemphasis{new owner} of this animal and
an email will be send to them as well as the responsible/contact person informing them about this
transaction. To claim more than one animal it is possible to select the desired animals at the first coloumn.
Please use after selection the button \sphinxstylestrong{Claim all selected animals}. Again a new page opens to review the selection.
After submitting the responsible person(s) of the animals only get one email with their claimed animals.
To claim an animal from a fish group (coloumn \# \textgreater{} 1) it is nessecary to claim it individualy with the claim function of
the entry (last coloumn).

\begin{sphinxadmonition}{note}{Note:}
If more than one animal is available (coloumn \# \textgreater{} 1), the user can adjust the number they want to claim.
The remaining animals will still be available for claim. Because of uniqueness it’s only
possible to offer exactly one mouse per dataset. In contrast fishes can be offer in a group.
\end{sphinxadmonition}

\noindent\sphinxincludegraphics[width=0.990\linewidth]{{anishare_claim}.png}


\section{Organs}
\label{\detokenize{index:organs}}
There exists an individual page for animal organ share. It is very similar to the animal page,
however only individual organs are for offer. The entry at the column \sphinxcode{\sphinxupquote{Organ (used)}} indicates
all organs which can not be claimed. Also there is no availability period, but a day at
which the animal gets sacrificed. The person responsible for sacrifice the animal will be informed via
email if anybody claims some of the available organs. The entry will remain available to others (as
they might want to claim other organs).

Organ index view:

\noindent\sphinxincludegraphics[width=0.990\linewidth]{{organs_index}.png}

Organ claim view:

\noindent\sphinxincludegraphics[width=0.990\linewidth]{{organs_claim}.png}


\section{RSS Feed}
\label{\detokenize{index:rss-feed}}
An RSS feed containing the latest ten animals and organs is automatically generated and can be found at
\sphinxtitleref{/animals/feed}. Users can subscribe (Most email clients allow the subscription
to RSS feeds) to this feed to stay up-to-date with the animal catalogue. By clicking on a link in
the feed, they are directed to the claim page of the individual animal/organ.

\noindent\sphinxincludegraphics[width=0.600\linewidth]{{anishare_rss_feed}.png}


\chapter{Main animal manager tasks}
\label{\detokenize{index:main-animal-manager-tasks}}
An \sphinxstyleemphasis{animal manager} can add animals and organs to the database on two ways. First, it is possible to
add entrys manually. Secondly, it is possible to import an Excel sheet. At the FLI Jena there are two
databases to manage animals. Now the databases are not connected. To transfer more than one or two datasets
to anishare it’s recommend to use the export function of \sphinxtitleref{PyRAT} or \sphinxtitleref{tick@lab} and the import function of anishare.
To use the export/import process please read the topic \sphinxstylestrong{Animals import} or \sphinxstylestrong{Organs import}.

\noindent\sphinxincludegraphics[width=0.800\linewidth]{{admin_overview_manager}.png}


\section{Add Animals manually}
\label{\detokenize{index:add-animals-manually}}
Click on \sphinxcode{\sphinxupquote{Animals}} -\textgreater{} \sphinxcode{\sphinxupquote{Add}} to add an animal.

\noindent\sphinxincludegraphics[width=0.600\linewidth]{{admin_add_animal}.png}

All fields in bold \sphinxstylestrong{need} to be filled in, the others are optional.

After adding several animals, the main (index) view should look like this:

\noindent\sphinxincludegraphics[width=0.990\linewidth]{{admin_after_loaddata}.png}


\section{Animals import}
\label{\detokenize{index:animals-import}}

\subsection{From PyRAT}
\label{\detokenize{index:from-pyrat}}
First login to PyRAT and \sphinxstylestrong{switch to the english version} of PyRAT if it is no preset. Then select the animals which
should be import to anishare. Click on \sphinxcode{\sphinxupquote{QS}} (Quick Select) and activate the option \sphinxcode{\sphinxupquote{Export this list to Excel}}.
Push the button \sphinxcode{\sphinxupquote{Apply}}.

\noindent\sphinxincludegraphics[width=0.600\linewidth]{{pyrat_export}.png}

Now it’s important to select all mandatory fields:
\sphinxstylestrong{ID, Lab ID, Sex, Line / Strine (Name), Mutations, Date of birth, Responsible, License number, Building}

It’s possible to save the selected columns as a \sphinxcode{\sphinxupquote{Manage View}} for reuse (fold out \sphinxcode{\sphinxupquote{Manage View}} on the left side of the \sphinxcode{\sphinxupquote{File name}})

\noindent\sphinxincludegraphics[width=0.600\linewidth]{{pyrat_export_select_columns}.png}

After downloading the file it’s nessecary to edit the file because the coloumns \sphinxstylestrong{Animal type, Available from, Available to} are missing.
To simplify this process there are macros for LibreOffice and MS Office. The macros automatically add the missing coloumns and add the values
\sphinxcode{\sphinxupquote{mouse}} (Animal type), \sphinxcode{\sphinxupquote{Current Date}} (Available from), \sphinxcode{\sphinxupquote{Current Date + 14 days}} (Available to).
Please refeer to the \sphinxhref{https://anishare.leibniz-fli.de/macros/}{macro site} to downloading the macros and further informations.

After adding the missing coloumns the file can be save. Please use the .xlsx file format. Now go to the anishare admin interface to \sphinxcode{\sphinxupquote{Home › Animals › Animals}}
and click the button \sphinxcode{\sphinxupquote{IMPORT}} (above the filter). Select the file and choose the file format. Upload the file. After submitting all datasets will show to
the user if all requirements match.


\subsection{From \sphinxtitleref{tick@lab}}
\label{\detokenize{index:from-tick-lab}}
First login to \sphinxtitleref{Tick@lab} and open the population site. All visible entrys can be exported with the button \sphinxtitleref{Export to Excel}. It isn’t yet possible to export
only selected animals. Therefore use the filter option.

\noindent\sphinxincludegraphics[width=0.800\linewidth]{{tickatlab_export}.png}

To import the file it’s nessecary to do a lot of changes to the structure of the data. So please use the macro which do the changes automatic.
Please refeer to the \sphinxhref{https://anishare.leibniz-fli.de/macros/}{macro site} to downloading the macro and further informations.

After running the macro the file can be save. Please use the .xlsx file format. Now go to the anishare admin interface to \sphinxcode{\sphinxupquote{Home › Animals › Animals}}
and click the button \sphinxcode{\sphinxupquote{IMPORT}} (above the filter). Select the file and choose the file format. Upload the file. After submitting all datasets will show to
the user if all requirements match.


\section{Add Organs manually}
\label{\detokenize{index:add-organs-manually}}
Click on \sphinxcode{\sphinxupquote{Organs}} -\textgreater{} \sphinxcode{\sphinxupquote{Add}} to add an organ.

\noindent\sphinxincludegraphics[width=0.800\linewidth]{{admin_add_organ}.png}

All fields in bold \sphinxstylestrong{need} to be filled in, the others are optional.


\section{Organs import}
\label{\detokenize{index:organs-import}}

\subsection{From PyRAT}
\label{\detokenize{index:id2}}
First login to PyRAT and \sphinxstylestrong{switch to the english version} of PyRAT if it is no preset. Then select the animals which
should be import to anishare. Click on \sphinxcode{\sphinxupquote{QS}} (Quick Select) and activate the option \sphinxcode{\sphinxupquote{Export this list to Excel}}.
Push the button \sphinxcode{\sphinxupquote{Apply}}.

Now it’s important to select all mandatory fields:
\sphinxstylestrong{ID, Lab ID, Sex, Line / Strine (Name), Mutations, Date of birth, Responsible, License number, Building, Sacrifice date, Sacrifice method}

\noindent\sphinxincludegraphics[width=0.600\linewidth]{{pyrat_export_organs}.png}

It’s possible to save the selected columns as a \sphinxcode{\sphinxupquote{Manage View}} for reuse (fold out \sphinxcode{\sphinxupquote{Manage View}} on the left side of the \sphinxcode{\sphinxupquote{File name}})

After downloading the file it’s nessecary to edit the file because the coloumns \sphinxstylestrong{Animal type, Euthanasia performed by, Comment} are missing.
Furthermore the format of the coloumn \sphinxstylestrong{Sacrifice date} needs to be adapt.
To simplify this process there are macros for LibreOffice and MS Office. The macros automatically add the missing coloumns and add the value
\sphinxcode{\sphinxupquote{mouse}} at coloumn \sphinxstylestrong{Animal type}
Please refeer to the \sphinxhref{https://anishare.leibniz-fli.de/macros/}{macro site} to downloading the macros and further informations.

After running the macro it’s possible to fill out the coloumn \sphinxstylestrong{Sacrifice method} with one of the following entries:
\sphinxstylestrong{CO2, cervicale dislocation, decapitation, blood withdrawl, finale heart punction, overdose anaesthetics, other} and
the coloumn \sphinxstylestrong{Organ used} with the following entries:
\sphinxstylestrong{bladder, bone marrow, brain, genitals, heart, intestine, kidney, liver, lungs, other, spleen, stomatch}
It’s also possible to add this entries after uploading the file.

\begin{sphinxadmonition}{note}{Note:}
It’s possible to add more than one entry at the field \sphinxstylestrong{Organ used}. Please use a comma as seperator like \sphinxcode{\sphinxupquote{brain, bladder}}.
\end{sphinxadmonition}

Please save the Excel file as .xlsx file. Now go to the anishare admin interface to \sphinxcode{\sphinxupquote{Home › Animals › Animals}}
and click the button \sphinxcode{\sphinxupquote{IMPORT}} (above the filter). Select the file and choose the file format. Upload the file. After submitting all datasets will show to
the user if all requirements match.


\subsection{From \sphinxtitleref{tick@lab}}
\label{\detokenize{index:id4}}
Because we expect only a small quantity of importing organs from \sphinxtitleref{tick@lab} it’s only possible to add entries manually.


\section{Duplicating entries}
\label{\detokenize{index:duplicating-entries}}
For input of multiple similar entries, it is possible to duplicate an animal or organ entry. For this, select one
or more entries in the list (see figure below) and select “\sphinxtitleref{copy animal}” from the dropdown menu and click
“\sphinxtitleref{Go}”.

\noindent\sphinxincludegraphics[width=0.750\linewidth]{{admin_copy_animal}.png}

Another option is to edit an existing animal and click on “\sphinxtitleref{Save as new}”. This will save the
currently edited animal as a new instance:

\noindent\sphinxincludegraphics[width=0.990\linewidth]{{admin_save_as_new}.png}


\chapter{Main administrator tasks}
\label{\detokenize{index:main-administrator-tasks}}
The administrator can edit more objects in the admin interface, namely not just animals and organs
but also labs, locations and persons:

\noindent\sphinxincludegraphics[width=0.600\linewidth]{{admin_overview}.png}


\section{Organs used}
\label{\detokenize{index:organs-used}}
These organs are standard values for the field \sphinxstylestrong{Organ used}.


\section{Animals}
\label{\detokenize{index:id5}}
The main category to administer are animals to share.
Here, several filters (such as “sex”, “location”, etc.) are available to search for any set of animals.

\noindent\sphinxincludegraphics[width=0.990\linewidth]{{admin_animals}.png}

\begin{sphinxadmonition}{note}{Note:}
in order to remove a claim (thus making the animal available again), either click on an animal
and remove the email address from the field “new owner”, or select one or multiple animals and
select the “clear claim” \sphinxstyleemphasis{Action} and click “Go”.
\end{sphinxadmonition}

\begin{sphinxadmonition}{note}{Note:}
Once created, an animal cannot be deleted, except by the administrator.
\end{sphinxadmonition}


\section{Labs}
\label{\detokenize{index:labs}}
Labs are research labs/research groups and need to have at least one responsible/contact person each

\noindent\sphinxincludegraphics[width=0.600\linewidth]{{admin_labs}.png}

\begin{sphinxadmonition}{note}{Note:}
Only \sphinxstyleemphasis{administrators} are allowed to see and change Labs
\end{sphinxadmonition}


\section{Locations}
\label{\detokenize{index:locations}}
Locations are where animals are stored. Usually something like room numbers or “animal house” or “fish facility”.

\noindent\sphinxincludegraphics[width=0.600\linewidth]{{admin_locations}.png}

\begin{sphinxadmonition}{note}{Note:}
Only \sphinxstyleemphasis{administrators} are allowed to see and change Locations
\end{sphinxadmonition}


\section{Persons}
\label{\detokenize{index:persons}}
Persons responsible for the animals. Could be a vet or similar.
Every animal needs to have a responsible person associated to them. This person then gets
an email when the animal is being claimed.

\noindent\sphinxincludegraphics[width=0.990\linewidth]{{admin_persons}.png}

\begin{sphinxadmonition}{note}{Note:}
Only \sphinxstyleemphasis{administrators} are allowed to see and change Persons
\end{sphinxadmonition}


\section{Make a user an animal manager}
\label{\detokenize{index:make-a-user-an-animal-manager}}
The \sphinxstyleemphasis{administrator} is also responsible for user/rights management.
In order to be able to add/edit animals, a user has to be in the group \sphinxstyleemphasis{animal manager} and have
\sphinxstyleemphasis{staff status} in the django admin interface. For this, an \sphinxstyleemphasis{administrator} has to go to the \sphinxhref{https://anishare.leibniz-fli.de/admin/auth/user/}{user management} in the admin interface by clicking “Home” -\textgreater{} “Authentication and
Authorization” -\textgreater{} “Users”. Here, they can make a \sphinxstyleemphasis{user} an \sphinxstyleemphasis{animal manager}, by setting these values (\sphinxstyleemphasis{staff}
and group \sphinxstyleemphasis{animal manager}):

\noindent\sphinxincludegraphics[width=0.750\linewidth]{{admin_permissions_user}.png}


\section{Anishare change history}
\label{\detokenize{index:anishare-change-history}}
New functions and bugfix at the system should be documented. So users can be informed about changes on the system.
All changes are visible to authenticated users. Please refeer to the site \sphinxhref{https://anishare.leibniz-fli.de/changehistory/}{AniShare Change History} to see all changes. Furthermore it’s possible to subscribe to
the \sphinxhref{https://anishare.leibniz-fli.de/changehistory/feed}{Anishare Version Feed} to stay informed.



\renewcommand{\indexname}{Index}
\printindex
\end{document}